\documentclass[12pt]{abnt}

\usepackage[utf8]{inputenc}
\usepackage[brazil]{babel}
\usepackage{graphicx}
\usepackage{color}
\usepackage[colorlinks,linkcolor=black,citecolor=black,urlcolor=black,plainpages=false,pagebackref,hyperfootnotes=false,pdfpagelabels]{hyperref}
%\usepackage[portugues,boxed,lined]{algorithm2e}

%Para inserir o códigos fonte
\usepackage{float}   
\usepackage{fancyvrb}

%===== Códigos Fonte =====
\newenvironment{codeverbatim}{\VerbatimEnvironment \small
   \begin{Verbatim}[xleftmargin=20mm]}
   {\end{Verbatim}}
%=======
\floatstyle{boxed}  % tipos: plain, boxed, ruled
\newfloat{codigo}{tbp}{lop}[section]  % numera os captions com  número de seção.
\floatname{codigo}{Código}
% nome para ser usado no sumário
\newcommand{\listofcodename}{Lista de Códigos}
%=========================

\title{``Atividade pártica de Segurança de Dados''}

\author{Anderson de França Queiroz\\
Tiago de França Queiroz}

%\date{Abril de 2012}

\begin{document}

\sloppy

\maketitle

\begin{titlepage}

  \vspace{6cm}

  \begin{flushleft}
    \sffamily\slshape
    ``If you don't stand for something,\\
	you'll fall for anything''\\
	(Filme Sucker Punch)
   
    \vspace{1cm}
    
    \end{flushleft}

\end{titlepage}

%\clearpage
\tableofcontents
\clearpage

\chapter{Descrição de um ataque}

Um atacante deseja tronar indisponível o acesso por SSH do servidor de um conhecido. O modo que ele escolhe para fazer é utilizar
um programa que ele desenvolveu para gerar requisições de conexção com o servidor initerruptamente partindo de vários computadores
diferentes, ou seja, um ataque de DDoS -- \textit{Distributed Denial-of-Service}.

A disponibilidade de serviço é muito importante para várias empresas, principalmente quando o produto que a empresa ofereçe é o serviço.
Ter um serviço pode causar perdas como: uma compra não ser realizada(sites de e-commerce); clientes trocarem de empresa para uma cujos
serviços não fiquem indisponíveis; perda de confiabilidade; entre outras coisas.

Um ataque simples
que visa indisponibilizar serviço é o ataque de negação de serviço, DoS (\textit{Denial-of-Service}), 
em resumo esse tipo de ataque realiza um número muito grande
de requisições ao serviço em curto espaço de tempo, assim o servidor não consegue reponder a todos, o que gera indisponibilidade do sistema.
Existe a variante distribuida do DoS, o DDoS (\textit{Distributed Denial-of-Service}), 
que utiliza simultaneamente vários computadores para realizar ataques de DoS a um mesmo alvo.


No cenário descrito o atacante desenvolve um programa que chama o cliente padrão de SSH do sistema operacional e tenta logar no host alvo. O programa
apenas solicita a conexão e utiliza uam senha qualquer, uma vez que o objetivo é apenas indisponibilizar o sistema e não invadi-lo.
De posse do programa ele vai a um laboratório de informica de sua universidade e instala o programa em todos os computadores de modo que
quando se faça o login o programa inicialize em background. Como é preciso apenas a senha de usuário para realizar essa operação e todos
os alunos utulizam o mesmo usuário, sempre que alguém logar no computador o ataque de DoS iniciará.

\chapter{Roteiro de ataque de DDoS}

O ataque de DDoS (\textit{Distributed Denial-of-Service}) que será estudado nesta aula objetiva indisponibilizar o serviço de SSH comumente
utulizado para acesso remoto a computadores. bla bla bla...

\section{Criando um programa de DoS}

A negação de serviço consiste em muitas requisições em um curto intervalo de tempo, de modo que o servidor não consiga atender a todas. Então nesse
experimento criaremos um programa em linguagem C que utiliza \textit{mult-thread} para realizar inúmereas requisições \texttt{SSH} a um servidor.

\begin{enumerate}

	\item Abra o editor ASCI de sua preferência, sugere-se a utulização do \texttt{VIM}. Em um terminal execute \texttt{vim};
	\item Digite o código REFERENCIAR O CODIGO;
	\item Salve com o nome \texttt{DoS.c}. Utilize ESC :w Dos.c;
	\item Saia o editor. Utilize ESC :q;
	\item Compile. No terminal execute \texttt{cc DoS.c -o DoS};

\end{enumerate}

%\renewcommand{\baselinestretch}{0.5}  % distância entre linhas
%\begin{codigo}[!hbt]
%   \tiny  % tamanho da fonte
%      %\vspace{2mm}
%      \VerbatimInput[xleftmargin=8mm,numbers=left,obeytabs=true]{Code/HelloWorldKernel.cu}
%   \caption{\textit{Hello World} com uma chamada de \textit{kernel}}
%   \label{code.HelloWorldKernel.cu}
%\end{codigo}


\clearpage
\bibliography{relatorio}

\end{document}
